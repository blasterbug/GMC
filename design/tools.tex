\Subsection{Tools}
	In this section a couple of different tools will be described.

	\Subsection{GIT}
		For version control and collaboration we will use Git along with Github
		to manage issue tracking and development history.
	\Subsection{Java RMI}
		Java RMI is used as a middleware to allow for remote method invocation and remote event notification.
		This solves the problem of distributed communication, whereas the task for Gcom is to produce a group communication middleware.

		Java RMI is a part of the standard Java APIs and is therefore easily accessable and well documented.
		We chose this over e.g. Corba as small experiments with Java RMI was
		successful and we decided to look no further.
	\Subsection{Maven}
		Maven is chosen as the build tool for the Gcom middleware.
		This provides both a build environment as well as it
		handles things like external dependencies and building
		large suits of software well while being easy to use.

	\Subsection{JUnit}
		JUnit is used together with Maven to allow for easy unit testing as 
		well as well integrated testing in the build environment.
		Other 

	\Subsection{IntelliJ}
		The main IDE that will be used is IntelliJ IDEA, as Maven and Git interfaces well with this amongst other things.

	\Subsection{Java Swing}
		The debug and example chat applications need a graphical user interface.
		This will be built with Java Swing as we have previous experience with this.

	\Subsection{Missing}
		Tools that we might need to look for during the project are listed here
		\begin{itemize}
			\item Integration testing
			\item Local simulations of multiple members (without having to test against members on physically remote machines.
			\item Mockup testing tools
		\end{itemize}
