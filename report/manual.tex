% Meta-info: Java version, build environment, RMI
\subsection{Environment}
\paragraph{}{
    To build the Gcom project, you need a working distribution of maven, java
 1.7 or above. Some scripts are available in the \texttt{bin}, assuming
 you use bash.
}


% Building JAR
\subsection{Compiling Gcom}
\paragraph{}{
    To use the Gcom artifact, you need to compile and package it. To do so,
 just run \texttt{\$ mvn package} in the Gcom folder. The sub-directory
 \texttt{target} may now contain two jar files, \texttt{Gcom-4.2.jar} and
 \texttt{Gcom-4.2-jar-with-dependencies.jar}. The first one can be imported
 to a new project and can be used as a library. The second jar file is the 
 project packed with all needed dependencies and can be use to run directly
 the name server needed by Gcom.
}

% Starting nameserver
\subsection{Name Server}
\paragraph{}{
    To r
}

% Demo chat-app description
\subsection{Gchat, the toy demo of Gcom}
\paragraph{}{
    To show off and prove Gcom works, we build a demonstration application,
 Gchat. It is basically a chat build on top of Gcom.
}

\begin{figure}[h]
    \begin{center}
        \includegraphics[scale=0.6]{figures/gchat_connection.png}
    \end{center}
    \caption{Connection window of gchat}
    \label{fig:gchat_connect}
\end{figure}

\paragraph{}{
    The application can be found in the sub-directory \texttt{Gchat}. 
 A script in the \texttt{bin} directory allow you to run the application easily
 by running \texttt{chat.sh}. The script need in the same order : a multicast 
 strategy, that defines the way the communication module will send messages to
 the other nodes, a ordering, that define the way messages will be processed and
 the host name (or the IP address where the name server runs. \newline 
 There are three different multicast strategies, the tree base (\texttt{tree}),
 which is not stable, the unrealible multicast (\texttt{unreliable}) and the
 reliable one (\texttt{reliable}). \newline
 As well, there are three different ordering way, \texttt{fifo}, which is the
 first in, first out way and need to be more tested. \texttt{causal} order and
 the \texttt{unordered} strategies. \newline
 If need, the application can be (re)compiled by giving \texttt{compile} as an
 argument of the application. Finally you can run the debug mode by adding 
 \texttt{debug} to the arguments.
}


\begin{figure}[h]
    \begin{center}
        \includegraphics[scale=0.4]{figures/gchat_chat.png}
    \end{center}
    \caption{Chat window of gchat}
    \label{fig:gchat_connectchat}
\end{figure}


% Starting client and creating or joining group

% Further dev 
	% Create own application using Gcom
	
\begin{figure}[h]
    \begin{center}
        \includegraphics[scale=0.5]{figures/debug_window.png}
    \end{center}
    \caption{Debug interface of Gcom}
    \label{fig:debugGui}
\end{figure}

% trouble shooting ?
\subsection{Trouble shooting}
\paragraph{}{
    The main problem you can get may concern the name server.
}

