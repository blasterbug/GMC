% Overview
	\texttt{Gcom} is built using a layered arcitecture with three layers; communication, ordering and group manangement as described in Figure \textbf{TODO}.
	The communication and ordering are completely independent while the group management depend on both of these to function, meaning they are easy to exchange.
	In all there are three entities in play in an instance of a Gcom application; Members, a leader and a nameserver.


	% What components are there?
		% Group members using com. and ordering, nameserver, leader


	% What purpose/responsibility do they have?

	% How do they communicate?
		% RMI

% Communication layer
	% General structure and design

	% Non-reliable

	% Reliable
	

% Ordering layer
	% General structure and design

	% Unordered

	% Causal

% Group management layer
	% General structure and design

	% Join/leave

	% Re-election
	\Subsubsection{Election}
		Gcom uses a straightforward and naive election algorithm.
		Whenever a member detects a crashed leader it elects itself and broadcasts this to the group and nameserver.
		This means that multiple new leaders can elect themselves at the same time, with the last one (in terms of real time) will come out as the leader.

	% Scalability, limitations and partitioning
