There are a couple of limitations to the Gcom middleware as well as the example Gchat applications.
This section covers some of them.

\Subsection{Heartbeat}
	One big limitation of the Gcom middleware is that it does not implement a feature for maintainence of the group, reporting and handling crashed members and similar problems.
	What is done in the current implementation is that whenever a member is found nonrespondant (e.g. when someone sends it a message) this member is reported as crashed and removed from everyones view.
	If it was the leader, an election is initiated.
	One problem with this is that an inactive group the leader can crash and render the group unavailable via the nameserver which in turn lets a joining member start a new group with the same name.
	A solution to this problem would be to implement a heartbeat that periodically polls the entire group for crashed members and handles accordingly.

\Subsection{Network partitioning}
	Network partitioning, when a network is split into more than one piece, is one of the more difficult problems to recover from.
	If this happens to a group in Gcom the two or more subgroups will continue to function but once the network is joined together it is currently not possible to reconstruct the original group correctly.

\Subsection{Multiple groups}
	A group communication middleware should allow for nodes to be members of more than one group at a time.
	There is preparation work done in Gcom for this but it is far from fully supported and not possible at the moment.

\Subsection{Nicknames}
	In the Gchat application, nicknames of everyone in the group aren't shown in the list of users but only displayed when they send a message.
